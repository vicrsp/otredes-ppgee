
%% bare_conf.tex 
%% V1.2
%% 2002/11/18
%% by Michael Shell
%% mshell@ece.gatech.edu
%% 
%% NOTE: This text file uses MS Windows line feed conventions. When (human)
%% reading this file on other platforms, you may have to use a text
%% editor that can handle lines terminated by the MS Windows line feed
%% characters (0x0D 0x0A).
%% 
%% This is a skeleton file demonstrating the use of IEEEtran.cls 
%% (requires IEEEtran.cls version 1.6b or later) with an IEEE conference paper.
%% 
%% Support sites:
%% http://www.ieee.org
%% and/or
%% http://www.ctan.org/tex-archive/macros/latex/contrib/supported/IEEEtran/ 
%%
%% This code is offered as-is - no warranty - user assumes all risk.
%% Free to use, distribute and modify.

% *** Authors should verify (and, if needed, correct) their LaTeX system  ***
% *** with the testflow diagnostic prior to trusting their LaTeX platform ***
% *** with production work. IEEE's font choices can trigger bugs that do  ***
% *** not appear when using other class files.                            ***
% Testflow can be obtained at:
% http://www.ctan.org/tex-archive/macros/latex/contrib/IEEEtran/testflow


% Note that the a4paper option is mainly intended so that authors in
% countries using A4 can easily print to A4 and see how their papers will
% look in print. Authors are encouraged to use U.S. letter paper when 
% submitting to IEEE. Use the testflow package mentioned above to verify
% correct handling of both paper sizes by the author's LaTeX system.
%
% Also note that the "draftcls" or "draftclsnofoot", not "draft", option
% should be used if it is desired that the figures are to be displayed in
% draft mode.
%
% This paper can be formatted using the peerreviewca
% (instead of conference) mode.
\documentclass[conference]{IEEEtran}
% If the IEEEtran.cls has not been installed into the LaTeX system files, 
% manually specify the path to it:
% \documentclass[conference]{../sty/IEEEtran} 
%\usepackage[brazil]{babel}
\usepackage[brazil]{babel}
\usepackage{amsmath}
\usepackage{multirow}
\usepackage[pdftex]{graphicx}
% some very useful LaTeX packages include:

%\usepackage{cite}      % Written by Donald Arseneau
% V1.6 and later of IEEEtran pre-defines the format
% of the cite.sty package \cite{} output to follow
% that of IEEE. Loading the cite package will
% result in citation numbers being automatically
% sorted and properly "ranged". i.e.,
% [1], [9], [2], [7], [5], [6]
% (without using cite.sty)
% will become:
% [1], [2], [5]--[7], [9] (using cite.sty)
% cite.sty's \cite will automatically add leading
% space, if needed. Use cite.sty's noadjust option
% (cite.sty V3.8 and later) if you want to turn this
% off. cite.sty is already installed on most LaTeX
% systems. The latest version can be obtained at:
% http://www.ctan.org/tex-archive/macros/latex/contrib/supported/cite/

%\usepackage{graphicx}  % Written by David Carlisle and Sebastian Rahtz
% Required if you want graphics, photos, etc.
% graphicx.sty is already installed on most LaTeX
% systems. The latest version and documentation can
% be obtained at:
% http://www.ctan.org/tex-archive/macros/latex/required/graphics/
% Another good source of documentation is "Using
% Imported Graphics in LaTeX2e" by Keith Reckdahl
% which can be found as esplatex.ps and epslatex.pdf
% at: http://www.ctan.org/tex-archive/info/
% NOTE: for dual use with latex and pdflatex, instead load graphicx like:
%\ifx\pdfoutput\undefined
%\usepackage{graphicx}
%\else
%\usepackage[pdftex]{graphicx}
%\fi

% However, be warned that pdflatex will require graphics to be in PDF
% (not EPS) format and will preclude the use of PostScript based LaTeX
% packages such as psfrag.sty and pstricks.sty. IEEE conferences typically
% allow PDF graphics (and hence pdfLaTeX). However, IEEE journals do not
% (yet) allow image formats other than EPS or TIFF. Therefore, authors of
% journal papers should use traditional LaTeX with EPS graphics.
%
% The path(s) to the graphics files can also be declared: e.g.,
% \graphicspath{{../eps/}{../ps/}}
% if the graphics files are not located in the same directory as the
% .tex file. This can be done in each branch of the conditional above
% (after graphicx is loaded) to handle the EPS and PDF cases separately.
% In this way, full path information will not have to be specified in
% each \includegraphics command.
%
% Note that, when switching from latex to pdflatex and vice-versa, the new
% compiler will have to be run twice to clear some warnings.


%\usepackage{psfrag}    % Written by Craig Barratt, Michael C. Grant,
% and David Carlisle
% This package allows you to substitute LaTeX
% commands for text in imported EPS graphic files.
% In this way, LaTeX symbols can be placed into
% graphics that have been generated by other
% applications. You must use latex->dvips->ps2pdf
% workflow (not direct pdf output from pdflatex) if
% you wish to use this capability because it works
% via some PostScript tricks. Alternatively, the
% graphics could be processed as separate files via
% psfrag and dvips, then converted to PDF for
% inclusion in the main file which uses pdflatex.
% Docs are in "The PSfrag System" by Michael C. Grant
% and David Carlisle. There is also some information 
% about using psfrag in "Using Imported Graphics in
% LaTeX2e" by Keith Reckdahl which documents the
% graphicx package (see above). The psfrag package
% and documentation can be obtained at:
% http://www.ctan.org/tex-archive/macros/latex/contrib/supported/psfrag/

%\usepackage{subfigure} % Written by Steven Douglas Cochran
% This package makes it easy to put subfigures
% in your figures. i.e., "figure 1a and 1b"
% Docs are in "Using Imported Graphics in LaTeX2e"
% by Keith Reckdahl which also documents the graphicx
% package (see above). subfigure.sty is already
% installed on most LaTeX systems. The latest version
% and documentation can be obtained at:
% http://www.ctan.org/tex-archive/macros/latex/contrib/supported/subfigure/

%\usepackage{url}       % Written by Donald Arseneau
% Provides better support for handling and breaking
% URLs. url.sty is already installed on most LaTeX
% systems. The latest version can be obtained at:
% http://www.ctan.org/tex-archive/macros/latex/contrib/other/misc/
% Read the url.sty source comments for usage information.

%\usepackage{stfloats}  % Written by Sigitas Tolusis
% Gives LaTeX2e the ability to do double column
% floats at the bottom of the page as well as the top.
% (e.g., "\begin{figure*}[!b]" is not normally
% possible in LaTeX2e). This is an invasive package
% which rewrites many portions of the LaTeX2e output
% routines. It may not work with other packages that
% modify the LaTeX2e output routine and/or with other
% versions of LaTeX. The latest version and
% documentation can be obtained at:
% http://www.ctan.org/tex-archive/macros/latex/contrib/supported/sttools/
% Documentation is contained in the stfloats.sty
% comments as well as in the presfull.pdf file.
% Do not use the stfloats baselinefloat ability as
% IEEE does not allow \baselineskip to stretch.
% Authors submitting work to the IEEE should note
% that IEEE rarely uses double column equations and
% that authors should try to avoid such use.
% Do not be tempted to use the cuted.sty or
% midfloat.sty package (by the same author) as IEEE
% does not format its papers in such ways.

%\usepackage{amsmath}   % From the American Mathematical Society
% A popular package that provides many helpful commands
% for dealing with mathematics. Note that the AMSmath
% package sets \interdisplaylinepenalty to 10000 thus
% preventing page breaks from occurring within multiline
% equations. Use:
%\interdisplaylinepenalty=2500
% after loading amsmath to restore such page breaks
% as IEEEtran.cls normally does. amsmath.sty is already
% installed on most LaTeX systems. The latest version
% and documentation can be obtained at:
% http://www.ctan.org/tex-archive/macros/latex/required/amslatex/math/



% Other popular packages for formatting tables and equations include:

%\usepackage{array}
% Frank Mittelbach's and David Carlisle's array.sty which improves the
% LaTeX2e array and tabular environments to provide better appearances and
% additional user controls. array.sty is already installed on most systems.
% The latest version and documentation can be obtained at:
% http://www.ctan.org/tex-archive/macros/latex/required/tools/

% Mark Wooding's extremely powerful MDW tools, especially mdwmath.sty and
% mdwtab.sty which are used to format equations and tables, respectively.
% The MDWtools set is already installed on most LaTeX systems. The lastest
% version and documentation is available at:
% http://www.ctan.org/tex-archive/macros/latex/contrib/supported/mdwtools/


% V1.6 of IEEEtran contains the IEEEeqnarray family of commands that can
% be used to generate multiline equations as well as matrices, tables, etc.


% Also of notable interest:

% Scott Pakin's eqparbox package for creating (automatically sized) equal
% width boxes. Available:
% http://www.ctan.org/tex-archive/macros/latex/contrib/supported/eqparbox/



% Notes on hyperref:
% IEEEtran.cls attempts to be compliant with the hyperref package, written
% by Heiko Oberdiek and Sebastian Rahtz, which provides hyperlinks within
% a document as well as an index for PDF files (produced via pdflatex).
% However, it is a tad difficult to properly interface LaTeX classes and
% packages with this (necessarily) complex and invasive package. It is
% recommended that hyperref not be used for work that is to be submitted
% to the IEEE. Users who wish to use hyperref *must* ensure that their
% hyperref version is 6.72u or later *and* IEEEtran.cls is version 1.6b 
% or later. The latest version of hyperref can be obtained at:
%
% http://www.ctan.org/tex-archive/macros/latex/contrib/supported/hyperref/
%
% Also, be aware that cite.sty (as of version 3.9, 11/2001) and hyperref.sty
% (as of version 6.72t, 2002/07/25) do not work optimally together.
% To mediate the differences between these two packages, IEEEtran.cls, as
% of v1.6b, predefines a command that fools hyperref into thinking that
% the natbib package is being used - causing it not to modify the existing
% citation commands, and allowing cite.sty to operate as normal. However,
% as a result, citation numbers will not be hyperlinked. Another side effect
% of this approach is that the natbib.sty package will not properly load
% under IEEEtran.cls. However, current versions of natbib are not capable
% of compressing and sorting citation numbers in IEEE's style - so this
% should not be an issue. If, for some strange reason, the user wants to
% load natbib.sty under IEEEtran.cls, the following code must be placed
% before natbib.sty can be loaded:
%
% \makeatletter
% \let\NAT@parse\undefined
% \makeatother
%
% Hyperref should be loaded differently depending on whether pdflatex
% or traditional latex is being used:
%
%\ifx\pdfoutput\undefined
%\usepackage[hypertex]{hyperref}
%\else
%\usepackage[pdftex,hypertexnames=false]{hyperref}
%\fi
%
% Pdflatex produces superior hyperref results and is the recommended
% compiler for such use.



% *** Do not adjust lengths that control margins, column widths, etc. ***
% *** Do not use packages that alter fonts (such as pslatex).         ***
% There should be no need to do such things with IEEEtran.cls V1.6 and later.


% correct bad hyphenation here
\hyphenation{op-tical net-works semi-conduc-tor IEEEtran}


\begin{document}
	
% paper title
\title{Programação de caminhões de minas de céu aberto minimizando o custo de manutenção}

	
% author names and affiliations
% use a multiple column layout for up to three different
% affiliations
\author{\authorblockN{Victor São Paulo Ruela}
	\authorblockA{Programa de Pós-Graduação em Engenharia Elétrica\\
		Universidade Federal de Minas Gerais\\
		Belo Horizonte, Brasil\\
		Email: victorspruela@gmail.com}}
	
% avoiding spaces at the end of the author lines is not a problem with
% conference papers because we don't use \thanks or \IEEEmembership


% for over three affiliations, or if they all won't fit within the width
% of the page, use this alternative format:
% 
%\author{\authorblockN{Michael Shell\authorrefmark{1},
%Homer Simpson\authorrefmark{2},
%James Kirk\authorrefmark{3}, 
%Montgomery Scott\authorrefmark{3} and
%Eldon Tyrell\authorrefmark{4}}
%\authorblockA{\authorrefmark{1}School of Electrical and Computer Engineering\\
%Georgia Institute of Technology,
%Atlanta, Georgia 30332--0250\\ Email: mshell@ece.gatech.edu}
%\authorblockA{\authorrefmark{2}Twentieth Century Fox, Springfield, USA\\
%Email: homer@thesimpsons.com}
%\authorblockA{\authorrefmark{3}Starfleet Academy, San Francisco, California 96678-2391\\
%Telephone: (800) 555--1212, Fax: (888) 555--1212}
%\authorblockA{\authorrefmark{4}Tyrell Inc., 123 Replicant Street, Los Angeles, California 90210--4321}}



% use only for invited papers
%\specialpapernotice{(Invited Paper)}

% make the title area
\maketitle

\begin{abstract}
	Este relatório apresenta os resultados do trabalho final da disciplina de Otimização em Redes 2020/1, o qual consiste na implementação de um modelo de programação linear inteiro proposto na literatura \cite{topal2010a} para a otimização da programação de equipamentos de mina de céu aberto. Utilizando a linguagem de programação Python e o solver CPLEX, a implementação foi realizada com sucesso e avaliada para diferentes instâncias do problema.
\end{abstract}

\section{Introdução}

O processo de mineração possui 5 etapas: prospecção, pesquisa, desenvolvimento,
exploração, reclamação \cite{GIFSThe599:online}. De acordo com \cite{newman2010}, otimização é principalmente utilizada nas etapas de esenvolvimento e exploração. Na etapa de desenvolvimento, o objetivo é o planejamento de como a mina deverá ser explorada ao longo dos anos, além da tomada de decisão de quais investimentos serão realizados. As principais aplicações são o planejamento da produção, e a seleção e alocação de equipamentos. Na etapa de exploração, o objetivo é otimizar os recursos disponíveis de forma a se atingir a produção planejada. Sistemas de despacho de caminhões são umas das suas principais
aplicações.

A mineração é um negócio de capital muito intenso, o qual requer um investimento da ordem de centenas de milhões de dólares somente para manter uma frota de equipamentos em operação. A manutenção da frota de caminhões é um dos maiores custos envolvidos, sendo responsável por 30-50\% dos custos de transporte de materiais em uma mina de céu aberto que utiliza frotas de caminhões e escavadeiras \cite{topal2010a}. Um investimento deste tamanho necessita que os equipamentos sejam utilizados da melhor forma ao longo do tempo, garantindo que os custos de operação sejam minimizados e a sua utilização maximizada, uma vez que pequenos ganhos podem resultar em uma economia na ordem de milhões. Logo, o uso de modelos de otimização possui grande potencial de aplicação neste contexto.

Em \cite{newman2010} é feita uma revisão das aplicações de pesquisa operacional no contexto de mineração, onde é constatado um grande interesse em problemas de programação inteira para otimização de diversos aspectos relacionados à frota da mina, visto o impacto que eles podem trazer para o negócio. Portanto, o objetivo deste trabalho será implementar o modelo de otimização da programação de caminhões para minas de céu aberto proposto por \cite{topal2010a}. 

Neste artigo, o autor propõe um modelo de programação linear inteiro visando minimizar os custos de manutenção e respeitando os objetivos de produção anuais. Ele considera a variação do custo de manutenção com a idade do caminhões por meio de faixas de idade, bem como a sua disponibilidade variável para movimentação de material em cada ano. A formulação possui alguns aspectos únicos, como o sequenciamento das faixas de idade utilizando variáveis binárias. Além disso, o modelo é capaz de considerar vários tipos de caminhões com diferentes idades iniciais, bem como otimizar globalmente uma frota completa para todo o ciclo de operação de uma mina.

Algumas variações desta formulação escolhida existem na literatura, as quais propõe principalmente a adição de novas restrições e diferentes objetivos, tornando cenários mais complexos de programação da frota possíveis. Em \cite{topal2010b}, os autores acrescentaram a opção de compra de caminhões novos ao longo do tempo. Já em \cite{cacetta2016}, uma formulação um pouco diferente é proposta onde é considerada a depreciação dos equipamentos de acordo com a sua idade. Já \cite{nakousi2018} desenvolveu um modelo mais completo que leva em consideração não só os custos de manutenção, mas também o consumo de combustível e a depreciação dos equipamentos. Portanto, nota-se que esse é um problema relevante na literatura, e com grandes oportunidades de continuidade.

\section{Metodologia}

\subsection{Formulação do problema}
O problema foi formulado com base na descrição disponível em \cite{topal2010a}, a qual foi adaptada de acordo com a implementação realizada. Inicialmente, são feitas as seguintes definições sobre os índices, parâmetros e variáveis de decisão do problema, conforme visto na Tabela \ref{tb:defs}, onde $t \in T=[1,\dots,t_{max}]$, $b \in B = [1,\dots,b_{max}]$ e $y \in Y = [1,\dots,y_{max}]$. A partir destas definições, o problema é formulado na sequência.


\begin{table}[h!]
	\begin{tabular}{p{1cm}p{2.5in}}
				\multicolumn{2}{l}{\textbf{Índices}}                                                            \\                           
		$t$                & Identificador do caminhão            \\
		$b$                & Faixas de idade           \\
		$y$                & Período de tempo (anos)   \\
		$c$                & Faixa de operação crítica \\ & \\
		\multicolumn{2}{l}{\textbf{Parâmetros}}                                                                                             \\
		$C_{t,b,y}$        & Custo de manutenção (\$/hora) 
		para um caminhão $t$ na faixa $b$ no $y$-ésimo período de tempo \\
		$FE_t$               & Custo de reparo do motor do caminhão $t$                                                     \\
		$A_{t,y}$          & Horas disponíveis do caminhão $t$ no período de tempo $y$                                      \\
		M                   & Tamanho da faixa de idade (horas)                                                          \\
		$R_y$                & Total de horas de operação para um período de tempo $y$ \\
		$I_t$ & Idade inicial dos caminhão $t$ \\
		&\\
		\multicolumn{2}{l}{\textbf{Variáveis de decisão}}                                                            \\
		$X_{t,b,y}$ & Número de horas alocadas para o caminhão $t$, faixa de idade $b$ no $y$-ésimo período de tempo \\
		$Y_{t,b,y}$ & 1, se o caminão $t$ na faixa $b$ utilizou todas as horas disponíveis no período de tempo $y$                                      
	\end{tabular}
	\label{tb:defs}
\end{table}



\begin{align}
  & \quad \text{\textbf{Minimizar}} \nonumber \\ \nonumber \\
  &  \sum_{y \in Y}^{}\sum_{t \in T}^{}\sum_{b \in B}^{}X_{t,b,y}C_{t,b,y} + \sum_{y \in Y}^{}\sum_{t \in T}^{}Y_{t,c,y}FE_t  \\ \nonumber \\
 & \quad \text{\textbf{Sujeito a}} \nonumber \\ \nonumber \\
 & \sum_{b \in B}^{}X_{t,b,y} \leq A_{t,y},\: \: \forall t \in T \:, \: \forall y\in Y \label{eq:r1}\\
 & \sum_{y \in Y}^{}X_{t,b,y} \leq M,\: \forall t \in T \:, \forall b \in B \label{eq:r2} \\
 & \sum_{k=1}^{y}X_{t,b,y} \geq MY_{t,b,y},\: \forall t \in T \:, \forall b \in B, \: \forall y\in Y \label{eq:r3} \\
 & X_{t,(b+1),y} \geq M \sum_{k=1}^{y}Y_{t,b,k},\: \forall t \in T \:, \forall b \in B, \: \forall y\in Y \label{eq:r4} \\
 & \sum_{t \in T}^{}\sum_{b \in B}^{}X_{t,b,y} = R_y ,\: \forall y \in Y \label{eq:r5} \\
 & \sum_{y \in Y}^{}\sum_{b \in B}^{}X_{t,b,y} \leq Mb_{max} - I_t ,\: \forall t \in T \label{eq:r6} \\
 & X_{t,b,y} \in \mathnormal{Z^+},\: \forall t \in T \:, \forall b \in B, \: \forall y\in Y \label{eq:r7} \\
 & Y_{t,b,y} \in {0,1},\: \forall t \in T \:, \forall b \in B, \: \forall y\in Y \label{eq:r8}
\end{align}

A restrição \ref{eq:r1} garante o respeito à disponibilidade de horas de cada caminhão, enquanto que \ref{eq:r2} não permite que horas alocadas ultrapassem o tamanho da faixa. As restrições \ref{eq:r3} e \ref{eq:r4} garantem que as faixas de idades sejam alteradas de forma sequencial. Já \ref{eq:r5} e \ref{eq:r6} garantem que o total de horas planejadas anualmente seja atingida e que os caminhões não sejam utilizados mais do que o valor máximo da faixa de idades, respectivamente. As restrições \ref{eq:r7} e \ref{eq:r8} definem o domínio das variáveis de decisão.

Isso resulta em um problema de programação linear inteiro, contendo somente variáveis de decisão inteiras e binárias. Levando em conta que o número de restrições e variáveis é bem alto, espera-se que este problema seja bem difícil de resolver em cenários de maior escala. 

\subsection{Geração das instâncias}

Conforme visto na seção anterior, existem diversos parâmetros que precisam ser definidos para a execução deste modelo. Analisando as informações contidas em \cite{topal2010a}, constatou-se que os dados de disponibilidade e custo de manutenção dos caminhões estão incompletos, de forma que a instância executada pelo autor não possa ser replicada.

Por esse motivo, foi proposto a verificação da disponibilidade destes dados de minas de céu aberto reais que utilizam sistmeas da \textit{Hexagon Mining}, os quais infelizmente também não estavam disponíveis. Então optou-se por gerar estes valores de forma artificial, seguindo as recomendações da literatura. Por outro lado, dados como a quantidade e disponibilidade de caminhões estao parcialmente disponíveis, podendo ser usados também como um guia para esta tarefa. A forma como estes dados são gerados é descrita a seguir.

\subsubsection{Disponibilidade dos caminhões}
De acordo com \cite{topal2010a}, esta disponibilidade é calculada com base no seu fator de utilização efetivo. Em um cenário hipotético onde o caminhão seria operado sem parar por um ano inteiro, teríamos um valor máximo de 8760 (465 * 24) horas disponíveis, resultando num fator com valor de 1. Como a manutenção do equipamento é necessária na prática, e assumindo diariamente são gastas 2 horas para a sua manutenção, teríamos 8395 horas disponíveis anualmente, o que resulta em um fator de utilização de aproximadamente 91.6\%. Como podem ocorrer flutuações nesse valor, os valores de disponibilidade serão amostrados de uma distribuição uniforme para um fator de disponibilidade entre 0.9 e 0.95 do total de horas anuais, conforme a Equação \ref{eq:disp}.

\begin{equation}
	A_{t,y} \sim \mathcal{U}_{[0.9, 0.95]}. 8760
	\label{eq:disp}
\end{equation}

Em \cite{nakousi2018}, os autores sugerem que a disponbilidade dos caminhões varia em função da sua idade, decrescendo até a sua idade crítica, e em seguida retornando aos patamares iniciais após o seu reparo. Entretanto, como isso necessitaria da modificação do modelo descrito anteriormente, a mesma não será implementada neste trabalho.

\subsubsection{Produção anual}
Seguindo a mesma idéia da seção anterior, a produção anual máxima possível consiste em multiplicar a quantidade de horas no ano pelo número de caminhões disponíveis para operação. Como isso é um cenário irreal por diversos motivos, devido a limitações da própria disponibilidade de material para ser transportado na mina, por exemplo, estes valores serão amostrados de uma distribuição uniforme no intervalo de 70 a 80\% desta disponibilidade, conforme a Equação \ref{eq:prod}. 

\begin{equation}
	R_y \sim \mathcal{U}_{[0.7, 0.8]}. 8760. t_{max}
	\label{eq:prod}
\end{equation}

Além disso, \cite{nakousi2018} sugere que a produção da mina deve ser reduzida a partir do penúltimo ano de operação, o que também é observado na instância descrita em \cite{topal2010a}. De acordo com as faixas de valores informadas pelos autores, o penúltimo e último ano serão multiplicados por um fator de escala 0.5 e 0.1, respectivamente, quando o número de anos da instância for maior do que 3. Um exemplo pode ser visto na Figura \ref{fig:prod_targets}.Vale ressaltar que a escolha dessas faixas deve ser feita de forma a garantir que o problema seja factível. Ou seja, não podemos ter uma produção anual que não possa ser atingida com os caminhões disponíveis.

\begin{figure}[h!]
	\centering
	\includegraphics[width=1\linewidth]{prod_targets.png}
	\caption{Exemplo de produção anual}
	\label{fig:prod_targets}
\end{figure}




\subsubsection{Idade dos caminhões}
Embora fornecidos pelo autor de \cite{topal2010a}, por padrão os valores serão amostrados de uma distribuição uniforme no intervalo de 0 a 20000 horas.

\subsubsection{Custo de manutenção}
De acordo com \cite{nakousi2018}, os custos de manutenção variam de forma não-linear de acordo com a idade do caminhão. Além disso, elas tendem a aumentar até um ponto onde um grande reparo é necessário, o que é representado pela faixa de idade crítica, após a qual o custo tende aos mesmos patamares de quando o caminhão era novo. Observando os valores fornecidos em \cite{topal2010a, topal2010b}, serão geradas amostras na faixa de 15 a 235 (\$/hora).

Entranto, como os caminhões podem possuir idades iniciais distintas, os custos de manutenção precisarão ser ajustados, uma vez que a formulação do problema não considera a idade inicial para definir o preço atual do caminhão. Portanto, a matriz de custos gerada será ajustada de forma que caminhões com maior idade inicial tenham um custo maior, pois ele já inicia numa faixa de idade diferente. 

Como a alocação dos caminhões é definida pelo otimizador, uma curva distinta será gerada para cada ano seguindo este formato, de forma a levar em consideração variações anuais no preço. Como as faixas de idade são incrementadas sequencialmente, é fácil ver que isto não viola o fato de que, embora um caminhão possa ter uma idade inicial diferente a cada ano, o custo sempre será referente à faixa de idade em que ele está atualmente.

Para cada caminhão $t$ e ano $y$, o custo para uma determinada faixa de idade $b$ é descrito pela Equação \ref{eq:costs}. Os valores médios por faixa $\mu_{b}$ são gerados de forma igualmente espaçada dentro do intervalo $[5,15]$ e em seguida elevados ao quadrado, simulando a não-linearidade. Além disso, um valor de ajuste é amostrado de uma distribuição normal com média 0 e desvio padrão 10 é acrescido. Um exemplo de curva pode ser visto na Figura \ref{fig:custo_manut}.

\begin{equation}
C_{t,b,y} = (\mu_{b})^2 + \mathcal{N}(0,10)
\label{eq:costs}
\end{equation}

\begin{figure}[h!]
	\centering
	\includegraphics[width=1\linewidth]{custos_manutencao.png}
	\caption{Exemplo de curva de custos de manutenção em função da idade}
	\label{fig:custo_manut}
\end{figure}


\subsection{Método de solução}
Como foi formulado um problema de programação linear inteiro de larga escala e com muitas restrições, ele será resolvido utilizando o algoritmo \textit{Branch and Cut} através do CPLEX versão 12.10.0.0. O trabalho será implementado na linguagem de programação Python versão 3.7.7, sendo necessários os pacotes \textit{pandas}, \textit{numpy}, \textit{matplotlib}, \textit{seaborn} e \textit{docplex}. Instruções para a instalação e configuração da biblioteca \textit{docplex} estão disponíveis em \cite{docplex:online}. Os testes serão executado em um notebook com Intel Core i7 Quad Core 2.4GHz, 8GB de memória RAM.

\subsection{Desenho dos experimentos}

O algoritmo será avaliado sobre 4 instâncias distintas do problema, cada uma possuindo diferente tamanho, conforme descrito na Tabela \ref{tab:instances}. Dessa forma, é possível validar a implementação e observar o quão difícil se torna esse problema aumentando a quantidade de variáveis e restrições.

Na instância \textit{Artigo}, serão considerados os valores disponíveis em \cite{topal2010a} para objetivos de produção, número de caminhões e suas respectivas idades inicias. Os demais valores serão gerados conforme descrito na seção anterior. O número de faixas de idade será igual a 20 para todas as instâncias.

\begin{table}[h!]
	\caption{Valores de cada instância gerada}
	\label{tab:instances}
	\centering
	\begin{tabular}{l|c|c|c|c|}
		\cline{2-5}
		& \textbf{Pequena} & \textbf{Média} & \textbf{Grande} & \textbf{Artigo} \\ \hline
		\multicolumn{1}{|l|}{\textbf{Caminhões}} & 5                & 20             & 45              & 34              \\ \hline
		\multicolumn{1}{|l|}{\textbf{Anos}}      & 3                & 5              & 10              & 10              \\ \hline
		\multicolumn{1}{|l|}{\textbf{M}}         & 2000             & 4000           & 5000            & 5000            \\ \hline
		\multicolumn{1}{|l|}{\textbf{Idade crítica}}         & 16000             & 40000           & 75000            & 	75000            \\ \hline
	\end{tabular}
\end{table}

Os valores de cada instância serão persisistidos em arquivos de texto para garantir sua reproducibilidade, uma vez que os dados são gerados de forma aleatória. Como é utilizado um algoritmo exato, somente uma execução é necessária para cada instância. Serão utilizados como critério de parada um \textit{gap} relativo mínimo ou tempo máximo de execução, quando for aplicável.

\section{Resultados}

\subsection{Instância pequena}

Os valores de custo utilizados podem ser visto na Figura \ref{fig:small_costs}, onde é importante notar que eles foram gerados conforme descrito nas seções anteriores. Um total de 480 variáveis de decisão e 567 restrições foram criadas para esta instância e o esforço computacional foi bem baixo para a sua solução, uma vez que o CPLEX encontrou o ótimo global em menos de 0.5 segundos.

\begin{figure}[h!]
	\centering
	\includegraphics[width=1\linewidth]{small_costs.png}
	\caption{Custos de manutenção por caminhão para a instância pequena}
	\label{fig:small_costs}
\end{figure}

As idades iniciais foram definidas como [0, 0, 8000 e 8000], bem como uma produção de 25000 horas por ano. Portanto, isso consegue simular o cenário onde pelos menos 1 caminhão atingirá a idade crítica e comparar ao cenário onde possa ser vantajoso utilizar os caminhões mais velhos pois o custo é menor após o reparo do motor, verificando se a implementação está correta. 

Na Figura \ref{fig:small_solution_values}, que contém a alocação de horas escolhidas pelo otimizador em cada ano, é possível notar que pelo menos dois caminhões atingiriam a idade crítica. Como o custo após o reparo do motor é muito menor, os caminhões mais antigos forma utilizados ao máximo no primeiro ano, de forma que nos dois anos seguintes eles possam ser utilizados com um custo menor. 

Para os caminhões novos, observe que eles possuem praticamente a mesma alocação no primeiro ano, porém partir do segundo ano ela foi feita de forma intercalada. Isso aconteceu pois existem pequenas variações de custo entre os anos, como pode ser visto na Figura \ref{fig:small_costs}. A partir da Figura \ref{fig:small_solutions_bins}, é possível ver que a restrição de sequência na escolha das faixas está sendo respeitada corretamente, e também que caminhões com idade iniciais e custos similares tiveram uma alocação de horas acumuladas bem próxima. Tendo em vista os resultados apresentados, é possível constatar que a implementação do modelo foi realizada com sucesso.

%\begin{figure}[h!]
%	\centering
%	\includegraphics[width=1\linewidth]{small_solution_accumulated.png}
%	\caption{Idade dos caminhões ao longo dos anos. Os valores estão em múltiplos de 1000 horas.}
%	\label{fig:small_solution_accumulated}
%\end{figure}

\begin{figure}[h!]
	\centering
	\includegraphics[width=1\linewidth]{small_solution_values.png}
	\caption{Alocação de horas por caminhão em cada ano para a instância pequena}
	\label{fig:small_solution_values}
\end{figure}

\begin{figure}[h!]
	\centering
	\includegraphics[width=1\linewidth]{small_solutions_bins.png}
	\caption{Alocação de horas por caminhão em cada faixa de idade para a instância pequena}
	\label{fig:small_solutions_bins}
\end{figure}

\subsection{Instância média}

Um total de 4000 variáveis de decisão e 4425 restrições foram criadas para esta instância e o esforço computacional também foi bem baixo para a sua solução, uma vez que o CPLEX encontrou o ótimo global em cerca de 3 segundos. Para o problema deste tamanho, já fica mais difícil definir quais seriam os resultados esperados, então uma análise mais detalhada será omitida. É interessante observar que o otimizador optou por alocar um dos caminhões além da sua idade crítica, pois era vantajoso do ponto de vista de custo. Isso pode ser visto na Figura \ref{fig:average_solutions_bins}, enquanto que a alocação escolhida pelo otimizador está na Figura \ref{fig:average_solution_values}.

%\begin{figure}[h!]
%	\centering
%	\includegraphics[width=1\linewidth]{average_solution_accumulated.png}
%	\caption{Idade dos caminhões ao longo dos anos para a instância média. Os valores estão em múltiplos de 1000 horas.}
%	\label{fig:average_solution_accumulated}
%\end{figure}

\begin{figure}[h!]
	\centering
	\includegraphics[width=1\linewidth]{average_solution_values.png}
	\caption{Alocação de horas por caminhão em cada ano para a instância média. Os valores estão em múltiplos de 1000 horas.}
	\label{fig:average_solution_values}
\end{figure}

\begin{figure}[h!]
	\centering
	\includegraphics[width=1\linewidth]{average_solutions_bins.png}
	\caption{Alocação de horas por caminhão em cada faixa de idade para a instância média}
	\label{fig:average_solutions_bins}
\end{figure}


\subsection{Instância grande}

Um total de 18000 variáveis de decisão e 18955 restrições foram criadas para esta instância e o esforço computacional foi relativamente grande para a sua solução, uma vez que o CPLEX chegou ao gap relativo de 1\% em cerca de 3.5 minutos. Da mesma forma que para a instância média, uma análise detalhada da solução é muito difícil para um problema deste tamanho, a qual será omitida. É importante notar que alguns caminhões são alocados de forma bem próxima da idade máxima de 100000 horas, e que vários foram utilizados além da idade crítica como opção para reduzir os custos. Isso pode ser visto na Figura \ref{fig:large_solution_accumulated}, enquanto que a alocação escolhida pelo otimizador está na Figura \ref{fig:large_solution_values}.

\begin{figure}[h!]
	\centering
	\includegraphics[width=1\linewidth]{large_solution_accumulated.png}
	\caption{Idade dos caminhões ao longo dos anos. Os valores estão em múltiplos de 1000 horas.}
	\label{fig:large_solution_accumulated}
\end{figure}

\begin{figure}[h!]
	\centering
	\includegraphics[width=1\linewidth]{large_solution_values.png}
	\caption{Alocação de horas por caminhão em cada ano para a instância grande. Os valores estão em múltiplos de 1000 horas.}
	\label{fig:large_solution_values}
\end{figure}

Um segundo experimento foi realizado para verificar o quanto este resultado poderia ser melhorado, o que foi feito reduzindo o gap mínimo para 0.1\% e deixando o algoritmo executar por 30 minutos. A evolução do gap pode ser vista na Figura \ref{fig:large_gap_progress_v2}, onde é possível constatar que após o gap de 1\%, o algoritmo conseguiu melhorar muito pouco a solução, demorando mais de 20 minutos para chegar a um gap final de 0.68\%, bem longe do limite especificado. Isso mostra que a etapa de polimento das soluções é muito difícil para este problema.

\begin{figure}[h!]
	\centering
	\includegraphics[width=1\linewidth]{large_gap_progress_v2.png}
	\caption{Progresso do algoritmo para a instância grande}
	\label{fig:large_gap_progress_v2}
\end{figure}



%\begin{figure}[h!]
%	\centering
%	\includegraphics[width=1\linewidth]{large_solutions_bins.png}
%	\caption{Alocação de horas por caminhão em cada faixa de idade para a instância grande}
%	\label{fig:large_solutions_bins}
%\end{figure}


\subsection{Instância artigo}

Um total de 13600 variáveis de decisão e 14324 restrições foram criadas para esta instância e o esforço computacional foi muito grande para a sua solução, uma vez que o CPLEX executou durante 30 minutos para chegar a um gap relativo de 6.40\%. Observe que embora possua um tamanho menor que a instância grande, sua solução foi muito mais difícil. Isso se deve ao fato de que mais da metade dos caminhões possuem mais de 40000 horas, ou seja, rapidamente eles irão chegar próximo da idade crítica, de forma que as restrições sejam mais fortes devido à escassez de recursos. Conforme a Figura \ref{fig:paper_gap_progress}, após 400 segundos de execução o algoritmo consegue evoluir muito pouco a qualidade da solução até o limite de tempo estipulado. É interessante notar que o algoritmo demorou menos de 1 minuto para chegar num gap de cerca de 14\%.

\begin{figure}[h!]
	\centering
	\includegraphics[width=1\linewidth]{paper_gap_progress.png}
	\caption{Exemplo de curva de custos de manutenção em função da idade}
	\label{fig:paper_gap_progress}
\end{figure}

Como boa parte dos caminhões estão com idade avançada, espera-se que grande parte seja utilizada até a seu limite de horas de operação, o que pode ser visto na Figura \ref{fig:paper_solution_accumulated}, que exibe o total de horas acumuladas ao longo dos anos. A alocacação nas faixas de idade podem ser vistas na Figura \ref{fig:paper_solutions_bins}. Contrariando um pouco a lógica, o algoritmo encontrou ser mais barato realizar o reparo do motor de 20 caminhões ao invés de utilizar ao máximo os mais novos. Isso pode ter sido causado pela característica de crescimento quadrático dos custos nas faixas de idade próximas da crítica. Portanto, pequenas alterações na solução pode levar a um aumento grande do objetivo, o que pode explicar a lentidão da evolução do algoritmo. Isso leva a crer que o uso de alguma meta-heurística neste ponto seja interessante.

Alguns parâmetros do CPLEX foram ajustados como forma de tentar melhorar o desempenho do algoritmo. Como houve pouco progresso após determinado tempo, a estratégia de \textit{probing} foi ajustada para o nível mais agressivo, o parâmetro de ênfase alterado para priorizar a otimalidade e a heurística de busca local foi ativada, conforme recomendação da IBM e as observações anteriores. Executando o algoritmo por 30 minutos, foi possível chegar a um gap de 5.76\%, o que mostra o ajuste de parâmetros tem potencial para melhorar o desempenho computacional obtido, embora tenha se chegado a uma solução não muito melhor neste teste. Como o autor de \cite{topal2010a} reportou um esforço computacional 30 minutos para atingir um gap de 5\%, os resultados estão coerentes com a literatura, embora os dados de custo e disponibilidade foram distintos.

\begin{figure}[h!]
	\centering
	\includegraphics[width=1\linewidth]{paper_solution_accumulated.png}
	\caption{Idade dos caminhões ao longo dos anos. Os valores estão em múltiplos de 1000 horas.}
	\label{fig:paper_solution_accumulated}
\end{figure}

%\begin{figure}[h!]
%	\centering
%	\includegraphics[width=1\linewidth]{paper_solution_values.png}
%	\caption{Alocação de horas por caminhão em cada ano. Os valores estão em múltiplos de 1000 horas.}
%	\label{fig:paper_solution_values}
%\end{figure}



\begin{figure}[h!]
	\centering
	\includegraphics[width=1\linewidth]{paper_solutions_bins.png}
	\caption{Alocação de horas por caminhão em cada faixa de idade para a instância grande}
	\label{fig:paper_solutions_bins}
\end{figure}


\section{Conclusão}

Neste trabalho for implementado um modelo de programação linear inteiro para otimização da programação de equipamentos de mina de céu aberto proposto na literatura \cite{topal2010a}. Utilizando o solver CPLEX e dados gerados artificialmente ou extraídos das referências utilizadas, foi possível analisar as soluções encontradas pelo algoritmo para 4 instâncias diferentes do problema. Por meio delas, foi possível constatar o quanto o problema fica mais difícil de acordo com o tamanho da instância, e também em relação aos dados considerados de custo, idades iniciais e disponibilidade. Para as 3 instâncias artificias, o CPLEX não teve grandes dificuldades para a sua solução. Entretanto, para dados mais próximos da realidade, foi notável o quão mais difícil o problema se tornou.
 
A maior dificuldade do trabalho consistiu na geração dos dados artificias e no entendimento das equações da formulação disponível na literatura. Até a obtenção de uma implementação que respondia de acordo com o esperado, diversos problemas ocorreram, dentre eles falta de respeito à ordem das faixas de idade e dados de custo inconsistentes com a idade inicial dos caminhões, por exemplo. Além disso, visualizar um problema onde cada variável de decisão possuía 3 índices não foi muito trivial, mas após o correto entendimento de como isso poderia ser feito, avaliar os resultados do algoritmo se tornou bem mais eficiente.

Todos os códigos utilizados no desenvolvimento deste trabalho também estão disponíveis online no repositório https://github.com/vicrsp/otredes-ppgee.


% harvard style
\begin{thebibliography}{1}

\bibitem{topal2010a}
Topal, E. and Ramazan, S., 2010. A new MIP model for mine equipment scheduling by minimizing maintenance cost. European Journal of Operational Research, 207(2), pp.1065-1071.

\bibitem{topal2010b}
Fu, Z., Topal, E. and Erten, O., 2014. Optimisation of a mixed truck fleet schedule through a mathematical model considering a new truck-purchase option. Mining Technology, 123(1), pp.30-35.

\bibitem{nakousi2018}
Nakousi, C., Pascual, R., Anani, A., Kristjanpoller, F. and Lillo, P., 2018. An asset-management oriented methodology for mine haul-fleet usage scheduling. Reliability Engineering \& System Safety, 180, pp.336-344.

\bibitem{docplex:online}
IBM® Decision Optimization CPLEX® Modeling for Python (DOcplex) V2.15 documentation: http://ibmdecisionoptimization.github.io/docplex-doc/. Acessado em  27/09/2020

\bibitem{GIFSThe599:online}
The 5 Stages of the Mining Life Cycle | Supply Chain \& Operations | Mining Global. https://www.miningglobal.com/operations/gifs-5-stages-mining-life-cycle. Acessado em 06/10/2020

\bibitem{newman2010}
Newman, Alexandra M., Enrique Rubio, Rodrigo Caro, Andrés Weintraub, and Kelly Eurek. "A
review of operations research in mine planning." Interfaces 40, no. 3 (2010): 222-245.

\bibitem{cacetta2016}
Burt, C., Caccetta, L., Fouché, L. and Welgama, P., 2016. An MILP approach to multi-location, multi-period equipment selection for surface mining with case studies. Journal of Industrial \& Management Optimization, 12(2), p.403.


\end{thebibliography}
	
	
	% that's all folks
\end{document}


